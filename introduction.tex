\section{Introduction}

Imagine if women bought dresses\footnote{To reverse the unavoidably gendered
nature of this example substitute men for women and suits for dresses.} the
way we currently buy smartphones. After waiting outside a glassy store, they
rush in and find a stack of the latest dress: the iDress~5. None is marked
with a size, and there are no dressing rooms, but this is the latest fashion
and so they proceed to checkout. At home they realize that every iDress is a
single size and few fit properly, but no tailor will alter them. When they
try to return it they are reminded that they agreed to wear it for two years.

Of course this is not how dresses are bought and sold. Fashion houses study
the human body to design shapely dresses in appropriate sizes. Women know
their measurements and select dresses by trying them on. And tailoring can
alter the fit of a dress to better match its owner. We argue that smartphones
are as personal as a dress and that the current ``one size fits all''
approach reduces performance and battery life. As an alternative we present a
measurement, design, selection and tailoring process we call
\textit{smartphone fitting} that ensures every smartphone user finds the
device that is a perfect fit: the myPhone, rather than the iPhone.

\section{Motivation}

\begin{figure}[t]
\includegraphics[width=\columnwidth]{./figures/flow.pdf}

\caption{\textbf{Overview of smartphone fitting.} Traces collected from users
are used both to guide device design, help users select devices, and continue
the process of fine-tuning device configuration during use.}

\label{figure-flow}
\end{figure}

Steve Jobs famously introduced the iPhone as three revolutionary
products---iPod, mobile phone, and Internet communicator---merged into
one~\cite{ipod-introduction-video}. So it should be no surprise that
smartphone usage varies considerably between users. Some play games, others
don't. Some use it mainly as a phone, others as an Internet communication
device. Some listen to a lot of music, others not so much. Some install lots
of apps, others few. Some charge only at night, others regularly during the
day. Some carry it in their purse, others in their pants.

\clearpage

Figures~\ref{figure-textvcall}~and~\ref{figure-breakdown} show evidence of
this variation from an experiment run on \PhoneLab{}, a 190~smartphone
programmable testbed at SUNY Buffalo~\cite{phonelab-web}.
Figure~\ref{figure-breakdown} shows the energy consumption for 77~\PhoneLab{}
participants\footnote{Not all 190~participants joined this experiment.} for a
single day, broken down into the categories recorded by Android. Differences
in energy consumption are significant and clearly reflect usage variation. As
a direct measure of usage differences, Figure~\ref{figure-textvcall} shows
phone and texting usage by participants for a single day and confirms the
folk wisdom that some users never use the phone but send hundreds of text
messages, while others never text but place multiple calls.

Differences in usage result in different hardware requirements, which cannot
be satisfied by a single device. Strangely, despite the fact that smartphones
are a more flexible technology than laptops and desktops, smartphone vendors
offer fewer options. As an example, Apple offers dozens of iMac desktop
configurations including choices of processors, memory, and peripherals, but
only three iPhone~5 configurations with differing amounts of storage. One of
the reasons for this may be that there is a lack of tools for guiding users
to the appropriate smartphone model, given that it is difficult if not
impossible for most users to determine how their needs map down to the
hardware configurations offered by vendors. Unfortunately, flexible
battery-constrained devices like smartphones suffer more from mismatches
between hardware and usage: underprovisioned components act as constraints,
while overprovisioned components drain unnecessary power. Smartphone users
need more options and more help when choosing.

\begin{figure}[t]
\includegraphics[width=\columnwidth]{./figures/textvcall.pdf}

\caption{\textbf{Texters, callers, and the more flexible.} Roughly equal
groups of \PhoneLab{} participants text exclusively, call exclusively, or do
both.}

\label{figure-textvcall}
\end{figure}
