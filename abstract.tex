\begin{abstract}

Electronic waste is a growing problem as the rapid pace of technological
improvements drives consumer appetites for the latest and greatest devices.
The result is millions of tons of e-waste, much of it containing hazardous
chemicals and difficult to dispose of safely. Smartphones are already part of
this problem, and given the rate of progress in the smartphone technologies,
it seems reasonable to expect that consumers will dispose of these devices at
rapid rates, spurred on by new features and discounts offered by carriers.

Fortunately, the capabilities, connectedness, and platform homogeneity of
smartphones make discarded devices ideal building blocks for many second
uses. Instead of ending up in a landfill, a discarded smartphone could be
transformed into part of a home security system or into health care device
for the elderly or disabled. In this paper, we propose using discarded
smartphones to replace traditional sensor network ``motes''. When compared
with motes, discarded phones have many advantages, including price, human and
sensor interfaces, ease of programming, performance and connectivity. The
main challenge is reducing their energy consumption to levels where energy
harvesting solutions can allow continuous operation, and we describe
preliminary results indicating that this may be possible.

\end{abstract}
