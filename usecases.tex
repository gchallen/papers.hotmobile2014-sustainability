\section{Use Cases}
\label{sec-usecases}
We describe two sensing applications using discarded smartphones that we are
currently investigating.
\subsection{Rooftop Monitoring}
\begin{figure}[t]
  \centering
  \subfigure[Rooftop Monitoring]{
    \includegraphics[width=0.45\columnwidth]{./figures/mocksetup.pdf}
  \label{fig-solarsetup}}\quad
  \subfigure[Urban Monitoring]{
    \includegraphics[width=0.45\columnwidth]{./figures/urban.pdf}
  \label{fig-urbanmonitor}}

  %\includegraphics[scale=0.1]{./figures/mocksetup.pdf}

  \vspace*{-0.1in}

  \caption{\small Sensing applications.
  \textnormal{Two of the sensing applications we are currently investigating.}}

  \vspace*{-0.1in}

  %\label{fig-solarsetup}
\end{figure}
Building rooftops can be equipped with discarded smartphones to monitor the
environmental conditions. This in turn can be used to regulate the lighting and
temperature inside the buildings to minimize the energy costs.
Due to the unavailability of power sources on rooftops and hazardous nature of
power extension cables, it is unviable to deploy smartphones as monitoring
instruments without any external power supplies.
Based on our experiments, we think this can be made possible by using other
alternate renewable energy sources like solar energy.

Figure~\ref{fig-solarsetup} shows our current deployment. The solar panel trickle
charges the smartphone battery. The application running in the smartphone
periodically senses the temperature and light levels. It then transmits the
sensed values to a central server every \textit{transmit} interval, a
configurable parameter, using 3G. 

Our initial measurements indicates that such a setup is
enough for a self sustaining monitoring application. The total cost for the
deployment was \$XX excluding the cost for the smartphone and data services.

\subsection{Urban Monitoring}
Cars manufactured post 1996 are equipped to provide On-Board diagnostics(OBD)
data following the OBD-II specification. With ODB-II, we can monitor rich
sensor data from the sensors deployed in the cars. With this data, we can design number of
applications like urban traffic monitoring, pollution monitoring ,etc. 

In order to read the sensor values from the car, we need to instrument the car
with a OBD
reader that has Bluetooth capability. We use the smartphones in the car to control
the OBD reader to retrieve the sensor data from the car's sensors. Once we have
the data, we can either transmit it immediately or store it to transmit at a
later time based on the application needs.

Since the smartphone is in the car, it can be powered using the car's power
sources like the cigarette ligther. We could also use the sensors on the
smartphone like the camera or the GPS to design smarter urban monitoring
applications without needing to worry of energy management for the smartphone
application.

The total cost for such a monitoring setup is around \$15 excluding the cost
for the smartphone and connectivity charges.

