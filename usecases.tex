\section{Use Cases}
\label{sec-usecases}
We describe two sensing applications using discarded smartphones that we are
currently investigating.
\subsection{Rooftop Monitoring.}
include setup figure
Building rooftops can be equipped with discarded smartphones to monitor the
environmental conditions. This in turn can be used to regulate the lighting and
temperature inside the buildings to minimize the energy costs.
Due to the unavailability of power sources on rooftops and hazardous nature of
power extension cables, it is unviable to deploy smartphones as monitoring
instruments without any external power supplies.
Based on our experiments, we think this can be made possible by using other
alternate renewable energy sources like solar energy.
Figure~\ref{solar-setup} shows our current deployment. The solar panel trickle
charges the smartphone battery. The application running in the smartphone
periodically senses the temperature and light levels. It then transmits the
sensed values to a central server every \textit{transmit} interval that is dynamically
configurable using 3G. Our initial measurements indicates that such a setup is
enough for a self sustaining monitoring setup. The total cost for the
deployment was \$XX excluding the cost for the smartphone and data services.
Purpose - 
 Environment monitoring
 roof top deployment on buildings
 no need external power sources due to extension cable fire hazards and laws.
 sense and send to central server
Describe setup 
 - solar panel charges the phone
 - phone sleeps, wakes up occasionally to sample sensor values.
 - periodically transmits collected sensor data using 3G.
 - goes back to sleep.
 - had to use Android version < 4.1 to enable airplane mode without the need of
 root privelages by the application.
sensors
  - temp
  - light
  - 
Sustainability-
  - Potentially no need of any intervention, if the application estimates the
  available energy efficiently.

Applications
 - 

\subsection{Urban Monitoring}
include figures from Immanuel
  Phones placed in the cars to monitor urban conditions.
  Feasible due to the density of cars in urban areas
  Setup
   - phone draws power from the car battery
    - direct connection to the battery or cig lighter
    - may be connected to other external sensors via BT (car CAN bus) or USB
  Sensors
   - Onboard 
    - GPS
   -External 
    - Car CAN bus
    - other sensors connected to the phone using usb host mode.

  Connectivity
   - either the phone has service enabled
   - uses the drivers phone connectivity via tethering
  
  Sustainability
  - afer initial setup and installtion in the car, no need of any more human
  involment.
  Applications
   - Enables spatio temporal queries of urban conditions

