\section{Preliminary Results}
\label{sec-results}

To explore the potential to transform our discarded Nexus~S~4G smartphones
into low-power sensors, the authors divided into two teams and engaged in a
lifetime programming competition. Our preliminary results were encouraging:
\XXXnote{FIXME}. We describe the application and two approaches to
programming it on Android devices in this section.

\subsection{A Lifetime Competition}

Each team was provided five discarded Nexus~S~4G phones and given
\XXXnote{FIXME} days to write a program that recorded battery and light
levels every 15~minutes and transmitted them to a server over a Wifi
connection. The goal was to implement a sensing application that would last
as long as possible, while maintaining data delivery to the server. A gap of
over two hours in the data values as observed by the other team rendered the
node as dead, regardless of the amount of energy it had reported, with the
two hour delay chosen to represent the potential requirements of a somewhat
delay-tolerant application.

Broadly speaking the teams both explored two different options with important
implications for reuse in this context: starting with the familiar Android
API and whittling the platform down to try to limit its inherent energy
usage, and discarding most of the Android platform and programming directly
against the underlying low-level APIs. We refer to the first approach as
``Sensor Android'' and the second as ``Sensor Nodroid''. As a baseline, we
also tried running one sensing application on an unmodified stock Android
image, unlinked to any user account and with no applications installed other
than those that come preinstalled with the Android Open-Source Platform.

From a programming perspective, we were hopeful that we could preserve the
familiar Android environment that many programmers today are learning. But
from an energy management persective, we were worried that the platform
represented too much of the typical short lifetime usage model of mobile
smartphone devices to support this unusual application. Further description
of the each approach follows.

\subsection{Sensor Android}

\XXXnote{GWA: Talk about what was necessary to discard a lot of the services,
etc., and what was left when we were done. How easy would it be to automate
this process?}

\subsection{Sensor Nodroid}

\XXXnote{GWA: Discuss what TinyAndroid is and what it preserves in the build,
as well as the difficulty in programming to the low-level APIs.}
