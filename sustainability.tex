\section{Sustainability Today}
\label{sec-sustainability}

We spoke to Sprint about their efforts in the area of smartphone
sustainability. Sprint has been recognized as one of the greenest companies
in the US by Newsweek's annual Green Rankings~\cite{sprintgreen-url}, and has
ambitious goals for greening mobile devices. Sprint offers users credit for
their old phones during the purchase of a new device, with the amount
depending on the phone model and condition. They aim to recover 90\% of their
users phones at end-of-life by 2017, an aggressive target given today's
industry average of recovering only 10\% and Sprint's current rate of 44\%.
Sprint's goal may also be difficult to achieve because many users choose to
retain their old phone as a spare device. And the user demand for new
smartphones shows no sign of abating, with multiple carriers offering new
plans tailored at users that want to replace their devices even more
frequently.

When we asked Sprint specifically about what they do with devices that they
repurchase, their answer focused on enabling reuse of smartphones as
smartphones. After paying to test and, if necessary, refurbish the returned
phone, they resell it as used to a second user, either in the US or overseas.
Of the phones they have recovered from their buyback program, 80\% can be
rebranded and reused, 10\% are desirable phones but not compatible with
Sprint's network, and the final 10\% are so broken that their only value is
the \$1.82 of gold they contain~\cite{cnn-goldinphone}.

While creating a market for used smartphones is an appropriate first step, it
ignores what happens when the second user returns the doubly-used but still
functional device. As with other electronics, the value of smartphones drops
extremely quickly. Sprint informed us that they were offering users only \$22
for a Samsung Nexus~S~4G in good condition, three years after it sold for
\$529 unlocked. After several iterations either one of two things will
happen: the buyback price will be too low to incentivize the user to return
the phone, or the phone will be old enough to not be attractive to users in
the market for a smartphone.

If a phone is too old or broken to be reused, Sprint first manually
disassembles the phones apart to recover valuable parts that can be reused
again---plastics, glass, batteries---prior to sending the phones for
recycling. Any part that cannot be reused is classified as e-waste. According
to Sprint, 1,180 metric tons of e-waste was collected for recycling in
2010~\cite{sprintpolicy-url}. The e-waste is recycled in recycling facilities
such as Sipi Metals that process scrap non-ferrous or precious metals. At
these facilities, the cell phones are first finely shredded and then smelted
to extract the precious metals such as gold, silver, palladium, and copper.
These metals are then captured into metal bars which are sold in the market
to be reused in other products.

Sprint tries to ensure that all of its recycling partners are environmentally
certified and that trans-boundary shipments of e-scrap from developed
countries to underdeveloped countries is prevented. Nevertheless, studies
have shown that a large amount of electronic waste continues to be shipped to
poor countries without regulations protecting the workers that dismantle it
or the environment~\cite{etbc-dumping}. 

\begin{comment}

% 16 Jan 2014 : SDH : Space

Part of the problem is that
system-on-chip smartphone designs are difficult to repair and dismantle, and
while the PhoneBloks project aims to create a modular ``phone worth
keeping''~\cite{phonebloks-url}, we believe that the form factor facilitated
by more integrated designs will continue to appeal to all but the most
environmentally-conscious consumers.

\end{comment}
