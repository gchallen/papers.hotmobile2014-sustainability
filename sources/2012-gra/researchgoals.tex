%          2. Research goals, including a problem statement
%              1. Problem statement
%              2. Outcome statement
%              3. Put work in context
%              “We are often asked to support work that would go ahead even
%              if we can't fund it, but we prefer to enable research that might otherwise not
%              be possible. To clarify which this is, please explain what funding you already
%              have for this area of research (none) and how the proposal relates to your
%              existing plans (discuss PIs research interests in this area). Do you plan to
%              build a capability for other research (1), provide a tool (2), reproduce a
%              prior result, collaborate with others to try something out, follow up on a
%              promising idea, or explore a new one? All are potentially of interest; we just
%              want to know.

\section{Goals}
\vspace*{-0.05in}
\begin{framed}
\textsc{The Problem:} \textbf{The progress of smartphone technologies
produces growing amounts of electronic waste.}
\end{framed}
\vspace*{-0.2in}

Smartphone technologies are advancing rapidly, bringing new power into users
pockets and changing the way that we live and work. The rapid rate at which
consumers purchase new smartphones can be seen as primarily a response to the
rate at which this technology is improving. Short device lifetimes, while
unfortunate from a sustainability perspective, help support companies that
build and sell smartphone hardware and software. Unfortunately smartphones,
like most other electronics, are difficult to dispose of properly. Many end
up in landfills or shipped to poor countries where they are dangerously
dismantled in an effort to collect precious materials.

Given the potential of the smartphone to bring about transformative
technological change, it becomes difficult to reduce \textit{demand} by
arguing that consumers should hang on to outdated devices in the name of
sustainability. Instead, we propose to focus on the \textit{supply} of fully-
or partially-functional outdated devices that society currently struggles to
put to use, and explore how this growing volume of techno-trash can be
efficiently reused.

There are three reasons why the time is right for this effort. First, unlike
previous generations of ``feature phones'', the current smartphone market is
coalescing around a small set of common platforms such as Android. This
platform homogeneity reduces the burden of supporting large numbers of
discarded devices. Second, current smartphones have an attractive feature set
for many non-phone applications: size and power requirements facilitating
easy deployment, microphones and cameras allowing them to double as sensors,
touch screens for interfacing with users.

Finally, smartphones are well-integrated into the existing communication
infrastructure. They can transmit data via text messages, over WiFi networks,
and via high-speed mobile communication technologies like 3G. If WiFi is
available, no service plans are required to allow recycled smartphones to
become part of the ``Internet of Things''. And with carriers increasingly
interested in ``machine-to-machine'' applications, we expect to see
increasing service flexibility allowing discarded devices to be cheaply
connected to pervasive mobile cellular and data networks.

To provide an idea of the potential of discarded devices, the U.S.
Environmental Protection Agency (EPA) estimated that 141~million mobile
devices became ready for end-of-life management in 2009, of which only 11.7
million (8\%) were collected for recycling~\cite{epa-ewasteweb}. The
129~million phones discarded in 2009 would be enough to place an average of
\textit{200 phones} on all 600,000~bridges in the United States, or every
\textit{2 feet} on every stretch of highway in the 46,876~mile interstate
highway system.

\vspace*{-0.05in}
\begin{framed}
\textsc{The Outcome:} \textbf{Several successful demonstrations of both
\textit{how} to transform discarded smartphones and \textit{what} to
transform them into.}
\end{framed}
\vspace*{-0.2in}

Working with a supply of ``discarded'' smartphones left over from a
previously-funded project, we will build several prototypes demonstrating how
to efficiently repurpose discarded phones. We currently have several ideas
for how to use recycled smartphones, including:

\begin{itemize}

\item At campus bus stops monitoring rider volumes and performing on-demand
bus routing.

%\item As campus safely systems, functioning as the basis of ``blue light''
%phones.

\item Bolted to stop lights helping improve traffic flow in congested
corridors.

%\item Facilitating new forms of classroom interaction in college courses.

\item At nearby farms tracking animal behavior.

\item Providing the basis for verifiable location services.

\item Fitted to rowing shells and providing data improving crew performance.

%\item At nearby parks tracking air quality and facility usage.

\item As special-purpose devices assisting the disabled or elderly.

\end{itemize}

The specific projects tackled will be a function of student interest and the
specifics of different opportunities that present themselves. But the goal is
to demonstrate two things conclusively. First, we want our case studies to
show that discarded smartphones \textbf{are still useful devices}. Second, we
want to demonstrate \textbf{how to use discarded phones as building blocks}
to enable new applications or lower the cost of existing ones. In contrast
with existing work~\cite{li-smartphonereuse}, which explored reusing phones
\textit{as phones}, we plan to focus on transforming smartphones into sensors
and actuators integrated with existing infrastructure.

Any code or tools produced will be open-sourced and made available for
collaborative development. In addition, short videos describing the process
and results of recycling smartphones will be posted on a project website for
public viewing. Finally, papers and posters will be presented at appropriate
venues including new forums emerging to share results on sustainable
computing.
