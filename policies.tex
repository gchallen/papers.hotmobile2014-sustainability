\section{Example Jouler Policies}

Jouler policies take energy usage and information delivery into account while
operating over the set of idealized components available on a specific
device. We expect that this cross-device energy management framework will
enable many new policies to be developed and shared between users. Users may
experiment with different policies to determine what works best for their
device, or be guided to an appropriate policy through tracing techniques or
usage characterization tools outside the scope of this proposal.

Energy management policies may prioritize energy usage over time, between
applications, or based on any other exogenous information such as the user's
current activity, location, or direct input. Below we provide examples of
several energy management policies that could be implemented in the Jouler
framework.

\begin{itemize}

\item \textbf{Charging pattern adaptation.} Preliminary results on
\PhoneLab{} show that users display varying charging habits. This policy
determines the users charging behavior and then adapts energy usage to try
and devote as much energy to improving application performance while ensuring
that the smartphone will survive until it is plugged in again. The policy may
also prompt periodic chargers based on their location to help them remember
to recharge when power is available. While smartphones are notorious for
running out of energy too soon, arriving at a plug with energy to spare is
also a problem, as that energy could have been used to improve performance.

\item \textbf{Gameplay optimization.} Smartphone games are a
rapidly-expanding segment of the application market. This policy identifies
gameplay and then optimizes the system by allowing the processor and memory
to consume more energy while slowing background tasks and components not
utilized by the game such as storage or networking.

\item \textbf{Rewarding efficient applications.} Not all applications are
implemented efficiently or utilized efficiently by users. This policy uses
the information delivery to energy usage ratio to shift energy towards
applications that are running efficiently. Note that this definition of
energy is appropriately user specific, as it depends on user interaction. An
application that performs a great deal of background caching may be
identified as efficient for a user that uses it often, where the caching pays
off, but simultaneously as inefficient for a user that uses it rarely where
the caching costs are wasted.

\item \textbf{Lower and lower.} This policy simply attempts to reduce
performance for each application until the user reports difficulty using it.
These settings are then saved and reused each time the application is run.
Alternatively, it could provide a kind of energy ``throttle'' allowing users
to speed up or slow down applications interactively at runtime.

\item \textbf{Maximum battery life.} In certain scenarios such as when users
are traveling the time to the next charging opportunity is unknown. In these
situations, all applications must be run as efficiently as possible, and this
can be enabled via a simple Jouler policy.

\end{itemize}

\textbf{By enabling these policies, Jouler facilitates a degree of control
and personalization of energy management not possible today.} We are excited
to see what new policies will emerge once such a policy framework is in
place, and believe that per-user energy management will help keep smartphones
in primary use for longer periods of time thereby improving smartphone
sustainability.
