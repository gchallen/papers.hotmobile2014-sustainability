\section{Other Forms of Reuse}
\label{sec-other}

\XXXnote{GWA: Thermostat, GPS unit, storage locker.}
Due to all the advantages we talked about above, including the richness of internal sensors, capability, and connectivity, it is important to recognize that discarded smartphones have more types of reuse than just being substitus for sensor nodes.


Most smartphones manufactured after the year 2008 are imbedded with GPS chips, which means that a discarded device has the potential to be a dedicated GPS unit and be a part of  a large-scale system. Vehicles' on-board navigator is a perfect example for this purpose. 


Thermostats are critical in some areas that are extremely cold in winter. Currently most thermostats in the market have no or very limited automacy, thus users always have to manually adjust the temperature, this also yields to energy efficiency problem.  Discarded smartphones solve this problem perfectly. First of all, discarded smartphones($22) are fairly cheap compared to dedicated thermostats($150 in average). Furthurmore, smartphones have strong processing power, compared to no or few computing power in dedicated thermostats, with a discarded smartphone, we can dynamically manage the temperature and energy comsumption of a house thus gain automacy and efficiency. On top of that, since the smartphones have connectivity, one can easily manage remotely via internet connection. 


Touch screen, in some people's mind, is the most attractive part of a smartphone. Making full use of its touch screen is also critical for reusing a smartphone. Many parts of daily life, like the storage locker, can ultilize this big screen to simplify people's work. With a touch screen, people can easily open up a locker without the pain by spinning around the padlock. 
\XXXnote{GWA: Fast array of discarded nodes?}

\section{Conclusions}
\label{sec-conclusion}
